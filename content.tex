\large
\textbf{Keyholder} wird/bleibt, wer

\begin{itemize}
	\item{für die gesamte Gruppe öffnen bzw. schließen will und nicht zum Eigenbedarf}
	\item{als Ansprechpartner für Mitglieder und andere Nutzer zur Verfügung steht}
	\item{die Verantwortung für den gesamten Space übernimmt und bei Bedarf das Hausrecht ausübt}
	\item{auf die Sicherheit von Nutzern und Inventar achtet}
	\item{sich um die weiteren Aufgaben für Keyholder kümmert (das beinhaltet das Delegieren von solchen, niemand kann alles allein schaffen)}
	\item{lang genug dabei und oft genug vor Ort ist, um einen Überblick über die Beteiligten sowie die Räume, Projekte und Maschinen und deren Besonderheiten zu behalten}
\end{itemize}

Die Nutzung des alten Post-Gebäudes ist wesentlich komplexer als noch in  der Raiffeisenstraße. Allein unser Bereich ist deutlich größer und  schwieriger zu überblicken als der, dann gibt es noch separate Räume im  Erdgeschoss und Keller, die nur über mit anderen Nutzern geteilte  Flächen, Treppenhäuser und Aufzüge erreicht werden können. Dazu kommen Besonderheiten der Elektrik (z.B. zwei Hauptversorgungen), schwierige Abfluss-Verhältnisse Richtung Post-Filiale, eine Sprinkler- und Brandmeldeanlage sowie Rettungs- und Fluchtwege mit Türen, die nicht einfach so geöffnet oder verschlossen werden dürfen (und Schließzylinder, deren Funktionen nicht sofort ersichtlich sind).

Der Zutritt zum hinteren Treppenhaus und den Fluren der Postbank im dritten Stock sowie zum Keller und insbesondere die Nutzung der Lastenaufzüge sind \textbf{Zusatzqualifikationen}, die mühsam erlernt werden müssen, wenn nicht schmerzhaft oder auf andere harte Tour erfahren werden soll, dass man Risiken unterschätzt hat. Also: wer nicht ausdrücklich zum Fahrstuhl-Ritter geschlagen wurde und in jedem Fall den Technik-Schlüssel nebst der Info, wie lange er die Katakomben aufsuchen will, bei einem wachen Auskenn-Keyholder lässt, bleibt im Space ;)

\textbf{Ob ein Mitglied Keyholder wird, entscheidet der BGB-Vorstand, da dieser im Problemfall auch für die Konsequenzen gerade stehen muss.}

\textit{Wenn Du Keyholder werden möchtest und eine Absage erhältst, nimm es bitte nicht persönlich - wir tun uns sicher nicht leicht damit. Absagen sind nicht in Stein gemeißelt. Es gibt Fälle, in denen nach einer Weile “Reife”, “Professionalisierung” oder “In-die-Gruppe-Wachsens” ein aktiver Keyholder für die Gruppe gewachsen ist.}

\begin{center}
	\textit{Der BGB-Vorstand - Stand: 15.04.2015}
\end{center}